% !TEX program = xelatex

\documentclass{muratcan_cv}

\setname{Ali}{Rahimi}
\setaddress{Esfahan/Iran}
\setmobile{+989135732644}
\setmail{alirahimii@protonmail.com}
\setposition{Work Student} %ignored for now
\setlinkedinaccount{https://www.linkedin.com/in/ali-rahimi-50b20999} %you can play with color of the template (red is also nice..)
\setgithubaccount{https://lrhm.github.io/react-portfolio} %you can play with color of the template (red is also nice..)
\setthemecolor{red} %you can play with color of the template (red is also nice..)

\begin{document}
%Set variables
%You can add sections, texts, explanations just by copying the style below. Replace the dummy texts "\lipsum[1][x-x]\par" with actual texts.
%Create header
\headerview
\vspace{1ex}
%Sections
%
% Summary
\addblocktext{Summary}{%
I'm a Software Engineer. Assign me with some tasks, I'll find the solutions and get it done.
Experienced in develpement of complex Android Applications with the Old (JAVA) and
Hot stacks such as Kotlin, JetPackComponents (MVVM/ROOM/DATABINDING/etc), Dependency Injection(Dagger), Dataflow(Retrofit/Moshi/Coroutines) and etc.
Always striving to use the right technology "Stacks" best fitted for the requirements of the project.
}
%
%Education
\section{Education} 
    \datedexperience{Isfahan Univercity of Technology}{2013-2019} 
    \explanation{B.Sc. in Software Engineering} 
     \explanationdetail{
     \coloredbullet\ 
     Course Work: Advanced Mathematics and Physics, Software Engineering Principals(Design, Architecture, OOP, etc) Data Structure, OS Architecture, Theory of Formal Languages(Compilers, Automata), Databases, Computer Networks, Software Engineering, IT Engineering, Statistics, AI, Computer Graphics, Economics, Microprocessors and etc.     }

     
     
     

%
% Experience
\section{Experience}
    %
    \datedexperience{PulseCo, Freelancer}{Sep 2014 - March 2015/ IR} 
    \explanation{Full front-end Android Developer} 
    \explanationdetail{\coloredbullet\ The very first application I wrote with Android SDK. Artistic Pantomime Game/Application with cool UX and UI written in Java for android 2.1 to 4.0. The APK still works perfectly in android 10 !}
    %
    \datedexperience{Velocity Inc.}{2015-Summer / Turkey} 
    \explanation{Intern as Developer/Tester} 
    %
    \datedexperience{Akbank}{2018-2019 / Turkey} 
    \explanation{Ios Developer} 
    \explanationdetail{\coloredbullet\ %
     \lipsum[1][1-2]\par %replace this part with actual text
     }
    %
    \datedexperience{Mobile-Software AG}{2019-2020 / Germany} 
    \explanation{Ios Developer} 
    \explanationdetail{\coloredbullet\ %
     \lipsum[1][4-5]\par %replace this part with actual text
     }
    %
    \datedexperience{BMW Autonomous Driving Campus}{2020-Now / Germany} 
    \explanation{Working Student} 
    \explanationdetail{\coloredbullet\ %
     \lipsum[1][3-4]\par %replace this part with actual text
     }
%
% Skills
\section{Skills}
    %
    \newcommand{\skillone}{\createskill{Programming Languages}{\textbf{\emph{Experienced:}} \ \  Python \cpshalf Swift \ \ \textbf{\emph{Familiar:}} \ \  Javascript \cpshalf Objective-C \cpshalf Bash \cpshalf Java \cpshalf Scheme}}
    %
    \newcommand{\skilltwo}{\createskill{Software Development}{Programming Paradigms \cpshalf GIT \cpshalf CLI \cpshalf Agile Methodology \cpshalf DevOps Lifecycles}}
    %
    \newcommand{\skillthree}{\createskill{Frameworks \ \& \ Libraries}{Jupyter \cpshalf Matplotplib \cpshalf Numpy \cpshalf Pandas \cpshalf Scikit-learn \cpshalf Gym \cpshalf PyTorch \cpshalf Tensorflow}}
    %
    \newcommand{\skillfour}{\createskill{iOS Programming}{RxSwift \cpshalf PromiseKit \cpshalf CocoaPods \cpshalf Autolayout/DSLs}}
    %
    \newcommand{\skillfive}{\createskill{Languages}{\textbf{\emph{Native:}} \ \  Turkish \ \ \textbf{\emph{Fluent:}} \ \ English \ \ \textbf{\emph{Beginner:}} \ \  German }}
    %
    \createskills{\skillone, \skilltwo, \skillthree, \skillfour, \skillfive}
%
% Experience
\section{Extra}
    \newcommand{\extraone}{%
    \lipsum[1][7-8]\par %replace this part with actual text
    }
    %
    \newcommand{\extratwo}{%
    \lipsum[1][9-10]\par %replace this part with actual text
    }
    %
    \newcommand{\extrathree}{%
    \lipsum[1][11-12]%replace this part with actual text
    }
    %
    \newcommand{\listofextras}{\extraone, \extratwo, \extrathree}
    %
    \createbullets{\listofextras}
%
%Footnote
\createfootnote
\end{document}
